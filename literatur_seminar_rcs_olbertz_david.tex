\documentclass[conference]{IEEEtran}
\IEEEoverridecommandlockouts
% The preceding line is only needed to identify funding in the first footnote. If that is unneeded, please comment it out.
%Template version as of 6/27/2024

\usepackage{cite}
\usepackage{amsmath,amssymb,amsfonts}
\usepackage{algorithmic}
\usepackage{graphicx}
\usepackage{textcomp}
\usepackage{xcolor}
\def\BibTeX{{\rm B\kern-.05em{\sc i\kern-.025em b}\kern-.08em
    T\kern-.1667em\lower.7ex\hbox{E}\kern-.125emX}}
\begin{document}

\title{Rich Communication Services\\als Nachfolger der SMS}

\author{\IEEEauthorblockN{David Olbertz}
    \IEEEauthorblockA{\textit{Hochschule Bonn-Rhein-Sieg} \\
        Sankt-Augustin, Germany \\
        davidolbertz@gmail.com}
}

\maketitle

\begin{abstract}
    Der Abstract wurde noch nicht verfasst.
\end{abstract}

\begin{IEEEkeywords}
    Diese, index, terms, existieren, noch, nicht
\end{IEEEkeywords}

\section{Einleitung}
Heutzutage ist das Smartphone ein wichtiger Bestandteil des Alltags. Es ermöglicht unter anderem die Kommunikation über große Distanzen hinweg, egal ob im privaten oder beruflichen Umfeld.

Zur Kommunikation via Text existiert der altbekannte Short Message Service (SMS), welcher es ermöglicht, einfache Kurznachrichten zu versenden. Allerdings wird diese Kommunikationsmethode mittlerweile kaum noch genutzt und wurde größtenteils durch Over-the-Top Instant Messenger, welche deutlich mehr Funktionalitäten bieten, wie Gruppenchats und das Teilen von Medien. Zusätzlich sind Nachrichten meist Ende-zu-Ende-Verschlüsselt und das Senden ist gebührenfrei.
Dieser Trend bewegte die GSM Association dazu, einen neuen moderneren Kommunikationsstandard zu entwickeln, welcher ähnliche Funktionen wie die Over-the-Top Instant Messenger bieten. Anders ist, dass es von Mobilfunkbetreibern anstatt von einem großen Unternehmen betrieben wird und dadurch auf fast allen Endgeräten verfügbar ist.

Ziel dieser Seminararbeit ist, Rich Communication Services als Nachfolger der SMS zu betrachten. Dazu werden zuerst die aktuell gängigen Technologien behandelt und im Anschluss darauf auf RCS im Detail eingegangen. Neben der Definition und Funktionsweise werden auch weitere Aspekte, wie die Vor- und Nachteile der RCS, dargestellt. Dabei werden auch die Eigenschaften von RCS mit denen von SMS verglichen, wobei besonders in heutigen Zeiten die Sicherheit eine große Rolle spielt.

\section{Entwicklung}

\subsection{Short Message Service}

\begin{itemize}
    \item seit den späten 1990ern im Einsatz
    \item Short Messaging Service Center (SMSC) routen Nachrichten
    \item Nachricht zwischen Base Station und Gerät verschlüsselt, innerhalb des Systems allerdings nicht
    \item External Short Message Entities
          \begin{itemize}
              \item managen das Senden \& Empfangen von viele Nachrichten für Unternehmen (z.B. für One-Time-Passwords)
          \end{itemize}
    \item "Gatekeeper" \& Interface zu SMS
    \item entweder direkten Zugriff auf SMSC über Short Message Peer-to-Peer (SMPP) oder Verkaufen Zugriff weiter (?)
\end{itemize}
\cite{sendoutsms}

\subsection{Over-the-Top Instant Messenger}

\begin{itemize}
    \item jeder Nutzer muss Client installieren
\end{itemize}
\cite{rcsmno}

\begin{itemize}
    \item hier eine Liste der Features von einem OTT Beispiel auflisten, wie z.B. WhatsApp
    \item Ende-zu-Ende-Verschlüsselung
    \item alles läuft über die Server des Messenger-Betreibers
\end{itemize}

\section{Rich Communication Services}

\subsection{Definition}

\begin{itemize}
    \item verschiedene Versionen nach und nach released
    \item erste Version nur enriched phonebook, enriched calling, enriched messaging
    \item phonebook
          \begin{itemize}
              \item Profil erstellen
              \item andere können Profil sehen (freiwillig, beide müssen Profil teilen)
          \end{itemize}
    \item calling
          \begin{itemize}
              \item Multimedia während dem Anruf teilen (Videoanruf)
          \end{itemize}
    \item messaging
          \begin{itemize}
              \item Dateitransfer
              \item Gruppenchats
          \end{itemize}
    \item bereits da Idee, dass es nativ im Betriebssystem integriert sein sollte
\end{itemize}
\cite{rcsuite}

\begin{itemize}
    \item seit dem Aufstieg von OTT sinkt die Nutzung von klassischen Kommunikationsarten wie SMS
    \item 2015 Nutzung in Deutschland bereits um 41 \% gesunken
\end{itemize}
\cite{ottmobinter}

\begin{itemize}
    \item nativ im Betriebssystem implementiert (von OS Entwickler oder OEM / Gerätehersteller)
    \item verschiedene Dienste miteinander verknüpft, wie Nachrichten schreiben, Kontakte-App, Telefon-App
    \item kann alternativ als App installiert werden (z.B. App der Telekom, ! brauche zusätzliche Quelle hier !)
\end{itemize}
\cite{uniprof}

\subsection{Funktionen}

\begin{itemize}
    \item "1-to-1 Messaging" - einfacher Nachrichtenaustausch zwischen zwei Personen
    \item Dateitransfer ist möglich (egal welcher Typ, Status: Pending, Progress, Cancelled, Sent, Delivered)
    \item bestimmte Dateien sollen auf bestimmte Art und Weise dargestellt werden (Video, Bild, Audio, GIF, PDF)
    \item man kann Sprachnachrichten aufnehmen und versenden
    \item man kann den aktuellen Status der Nachricht sehen (Versendet, Fehler, SMS-Fallback, Gelesen)
    \item man kann sehen, wenn der Kontakt am schreiben ist
    \item man kann seinen (Live-)Standort teilen
    \item Gruppenchats
    \item "Enriched Calling" - Dateitransfer während Anruf möglich $\rightarrow$ Videoanruf
    \item Chatbots für z.B. Unternehmen (Gemini Integrierung ! Quelle nötig !), enthalten Beschreibung
    \item (mit Chatbots kann anonym gechattet werden)
    \item Spam kann vermieden werden
\end{itemize}
\cite{uniprof}

\subsection{Technik}

\begin{itemize}
    \item erste version
    \item unified composer (entscheidet selbst, ob als SMS oder MMS gesendet wird und sagt user bescheid)
\end{itemize}
\cite{rcsuite}

\begin{itemize}
    \item damit alles nahtlos funktioniert, müssen die Mobilfunk-Operatoren sich auf eine gängige Implementierung einigen
    \item 2011 unter anderem Deutsche Telekom, Telefonica, Vodafone angekündigt, Zusammenschluss zu bilden für eine gemeinsame RCS Implementierung, auch RCS-e (oder Rich Communication Services enhanced) genannt
\end{itemize}
\cite{rcsmno}

\begin{itemize}
    \item "Capability Discovery" zeigt den Clients, welche RCS Funktionen verfügbar sind
    \item RCS Dienste werden automatisch aktiviert
    \item es wird dem Nutzer direkt angezeigt, ob RCS verfügbar ist und falls nicht, wird auf SMS als Backup zurückgegriffen
\end{itemize}
\cite{uniprof}

\subsection{Sicherheit}

\begin{itemize}
    \item zur Authentifizierung am Endgerät Sim-basierte Authentifizierung nutzen, auf anderen Geräten andere Möglichkeiten (Login mit Username \& Passwort / Pairing über Code)
    \item Verschlüsselung von Allem mit gängigen Protokollen, wie TLS und IPsec
\end{itemize}
\cite{uniprof}

\section{Vorteile von RCS gegenüber SMS}

SMS

\begin{itemize}
    \item Nachricht zwischen Base Station und Gerät verschlüsselt, innerhalb des Systems allerdings nicht
    \item Gateways speichern Nachrichten teilweise für längere Zeit
    \item External Short Message Entities können für Betrug genutzt werden
\end{itemize}
\cite{sendoutsms}

\begin{itemize}
    \item anstatt für jeden Instant Messenger eigenen Client zu installieren, der Leistung braucht, universeller integrierter Messenger
    \item es ist möglich, RCS über ein Mobilgerät oder auch PC zu nutzen
\end{itemize}
\cite{rcsmno}

\begin{itemize}
    \item man kann clients auf mehreren Geräten nutzen
    \item bei Dual-SIM ist RCS mit beiden Karten gleichzeitig möglich
\end{itemize}
\cite{uniprof}

\section{Schwächen von RCS}

\begin{itemize}
    \item SMS SIM Swap Attack
          \begin{itemize}
              \item Angreifer gibt sich als das Opfer aus, um neue SIM-Karte zu erhalten oder eine Umleitung einzurichten
              \item kann Nachrichten abfangen, wie One-Time-Passwords
              \item durch RCS nicht gelöst, da es auch über die Telefonnummer läuft
          \end{itemize}
\end{itemize}
\cite{sendoutsms}

\begin{itemize}
    \item alle Mobilfunkanbieter müssen zusammenarbeiten, damit die Kommunikation auch über Ländergrenzen hinweg nahtlos funktionieren kann, es sollte eine standardisierte Implementation geben
    \item alle Hersteller von Mobilgeräten müssen RCS implementieren, damit die Verbreitung besser vorangetrieben wird
\end{itemize}
\cite{rcsmno}

\begin{itemize}
    \item Samsung Messages
    \item Google Messages
    \item iMessage (nur neuste iOS Version)
\end{itemize}

\section{Fazit}

\dots

\begin{thebibliography}{00}
    \bibitem{sendoutsms} B. Reaves, N. Scaife, D. Tian, L. Blue, P. Traynor, und K. R. B. Butler, "Sending Out an SMS: Characterizing the Security of the SMS Ecosystem with Public Gateways", in 2016 IEEE Symposium on Security and Privacy (SP), Mai 2016, S. 339–356. doi: 10.1109/SP.2016.28.
    \bibitem{rcsuite} K. Henry, Q. Liu, und S. Pasquereau, "Rich Communication Suite", in 2009 13th International Conference on Intelligence in Next Generation Networks, Okt. 2009, S. 1–6. doi: 10.1109/ICIN.2009.5357089.
    \bibitem{rcsmno} M. Lin und J. Arenzana Arias, "Rich Communication Suite: The challenge and opportunity for MNOs", in 2011 15th International Conference on Intelligence in Next Generation Networks, Okt. 2011, S. 187–190. doi: 10.1109/ICIN.2011.6081071.
    \bibitem{ottmobinter} N. Wellmann, "Are OTT messaging and mobile telecommunication an interrelated market? An empirical analysis", Telecommun. Policy, Bd. 43, Nr. 9, Oktober 2019, doi: 10.1016/j.telpol.2019.101831.
    \bibitem{uniprof} GSMA. "RCS Universal Profile Service Definition Document". Zugriff: 06.11.2024. Verfügbar: https://www.gsma.com/solutions-and-impact/technologies/networks/wp-content/uploads/2019/10/RCC.71-v2.4.pdf
\end{thebibliography}

\end{document}