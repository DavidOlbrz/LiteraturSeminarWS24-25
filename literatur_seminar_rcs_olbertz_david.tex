\documentclass[conference]{IEEEtran}
\IEEEoverridecommandlockouts
% The preceding line is only needed to identify funding in the first footnote. If that is unneeded, please comment it out.
%Template version as of 6/27/2024

\usepackage{cite}
\usepackage{amsmath,amssymb,amsfonts}
\usepackage{algorithmic}
\usepackage{graphicx}
\usepackage{textcomp}
\usepackage{xcolor}
\def\BibTeX{{\rm B\kern-.05em{\sc i\kern-.025em b}\kern-.08em
    T\kern-.1667em\lower.7ex\hbox{E}\kern-.125emX}}
\begin{document}

\title{Rich Communication Services\\als Nachfolger der SMS}

\author{\IEEEauthorblockN{David Olbertz}
    \IEEEauthorblockA{\textit{Hochschule Bonn-Rhein-Sieg} \\
        Sankt-Augustin, Germany \\
        davidolbertz@gmail.com}
}

\maketitle

\begin{abstract}
    Der Abstract wurde noch nicht verfasst.
\end{abstract}

\begin{IEEEkeywords}
    Diese, index, terms, existieren, noch, nicht
\end{IEEEkeywords}

\section{Einleitung}
Heutzutage ist das Smartphone ein wichtiger Bestandteil des Alltags. Es ermöglicht unter anderem die Kommunikation über große Distanzen hinweg, egal ob im privaten oder beruflichen Umfeld.

Zur Kommunikation via Text existiert der altbekannte Short Message Service (SMS), welcher es ermöglicht, einfache Kurznachrichten zu versenden. Allerdings wird diese Kommunikationsmethode mittlerweile kaum noch genutzt und wurde größtenteils durch Over-the-Top Instant Messenger, welche deutlich mehr Funktionalitäten bieten, wie Gruppenchats und das Teilen von Medien. Zusätzlich sind Nachrichten meist Ende-zu-Ende-Verschlüsselt und das Senden ist gebührenfrei.
Dieser Trend bewegte die GSM Association dazu, einen neuen moderneren Kommunikationsstandard zu entwickeln, welcher ähnliche Funktionen wie die Over-the-Top Instant Messenger bieten. Anders ist, dass es von Mobilfunkbetreibern anstatt von einem großen Unternehmen betrieben wird und dadurch auf fast allen Endgeräten verfügbar ist.

Ziel dieser Seminararbeit ist, Rich Communication Services als Nachfolger der SMS zu betrachten. Dazu werden zuerst die aktuell gängigen Technologien behandelt und im Anschluss darauf auf RCS im Detail eingegangen. Neben der Definition und Funktionsweise werden auch weitere Aspekte, wie die Vor- und Nachteile der RCS, dargestellt. Dabei werden auch die Eigenschaften von RCS mit denen von SMS verglichen, wobei besonders in heutigen Zeiten die Sicherheit eine große Rolle spielt.

\section{Vorgänger / historische Entwicklung}

\subsection{Short Message Service}

\begin{itemize}
    \item seit den späten 1990ern im Einsatz
    \item Short Messaging Service Center (SMSC) routen Nachrichten
    \item Nachricht zwischen Base Station und Gerät verschlüsselt, innerhalb des Systems allerdings nicht
    \item External Short Message Entities
          \begin{itemize}
              \item managen das Senden \& Empfangen von viele Nachrichten für Unternehmen (z.B. für One-Time-Passwords)
          \end{itemize}
    \item "Gatekeeper" \& Interface zu SMS
    \item entweder direkten Zugriff auf SMSC über Short Message Peer-to-Peer (SMPP) oder Verkaufen Zugriff weiter (?)
\end{itemize}
\cite{sendoutsms}

\subsection{Over-the-Top Instant Messenger}

\dots

\section{Rich Communication Services}

\subsection{Definition}

\begin{itemize}
    \item verschiedene Versionen nach und nach released
    \item erste Version nur enriched phonebook, enriched calling, enriched messaging
    \item phonebook
          \begin{itemize}
              \item Profil erstellen
              \item andere können Profil sehen (freiwillig, beide müssen Profil teilen)
          \end{itemize}
    \item calling
          \begin{itemize}
              \item Multimedia während dem Anruf teilen (Videoanruf)
          \end{itemize}
    \item messaging
          \begin{itemize}
              \item Dateitransfer
              \item Gruppenchats
          \end{itemize}
    \item bereits da Idee, dass es nativ im Betriebssystem integriert sein sollte
\end{itemize}
\cite{rcsuite}

\subsection{Funktionsweise}

\begin{itemize}
    \item erste version
    \item unified composer (entscheidet selbst, ob als SMS oder MMS gesendet wird und sagt user bescheid)
\end{itemize}
\cite{rcsuite}

\subsection{Sicherheit}

\dots

\section{Vorteile von RCS gegenüber SMS}

SMS

\begin{itemize}
    \item Nachricht zwischen Base Station und Gerät verschlüsselt, innerhalb des Systems allerdings nicht
    \item Gateways speichern Nachrichten teilweise für längere Zeit
    \item External Short Message Entities können für Betrug genutzt werden
\end{itemize}
\cite{sendoutsms}

\section{Schwächen von RCS}

\begin{itemize}
    \item SMS SIM Swap Attack
          \begin{itemize}
              \item Angreifer gibt sich als das Opfer aus, um neue SIM-Karte zu erhalten oder eine Umleitung einzurichten
              \item kann Nachrichten abfangen, wie One-Time-Passwords
              \item durch RCS nicht gelöst, da es auch über die Telefonnummer läuft
          \end{itemize}
\end{itemize}
\cite{sendoutsms}

\section{Fazit}

\dots

\begin{thebibliography}{00}
    \bibitem{sendoutsms} B. Reaves, N. Scaife, D. Tian, L. Blue, P. Traynor, und K. R. B. Butler, "Sending Out an SMS: Characterizing the Security of the SMS Ecosystem with Public Gateways", in 2016 IEEE Symposium on Security and Privacy (SP), Mai 2016, S. 339–356. doi: 10.1109/SP.2016.28.
    \bibitem{rcsuite} K. Henry, Q. Liu, und S. Pasquereau, "Rich Communication Suite", in 2009 13th International Conference on Intelligence in Next Generation Networks, Okt. 2009, S. 1–6. doi: 10.1109/ICIN.2009.5357089.
    \bibitem{rcsmno} M. Lin und J. Arenzana Arias, "Rich Communication Suite: The challenge and opportunity for MNOs", in 2011 15th International Conference on Intelligence in Next Generation Networks, Okt. 2011, S. 187–190. doi: 10.1109/ICIN.2011.6081071.
    \bibitem{ottmobinter} N. Wellmann, "Are OTT messaging and mobile telecommunication an interrelated market? An empirical analysis", Telecommun. Policy, Bd. 43, Nr. 9, Oktober 2019, doi: 10.1016/j.telpol.2019.101831.
\end{thebibliography}

\end{document}